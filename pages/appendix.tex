\chapter{Appendix}

\section{Fast Multipole Method Algorithm}\label{app:a_1_fmm_algorithm}

We provide the pseudocode for the \textit{adaptive} algorithm. Here, adjacent nodes can be of different sizes, as they are in general only constrained by the geometry of the problem a the user defined parameter for the maximum number of particles per node. \gls{FMM} literature distinguish between different types of relationships that can exist between neighbouring nodes with the concept of \textit{interaction lists}. There are four lists for each box $B$ in a given tree, called the $V$, $U$, $W$ and $X$ interaction lists, respectively. For leaf box $B$, the $U$ list contains $B$ itself and leaf boxes adjacent to $B$. The $V$ list is the set of children of the neighbours of the parent of $B$ which are not adjacent to $B$. If $B$ is a leaf node, the $W$ list consists of the descendants of $B$'s neighbours whose parents are adjacent to $B$. For a non-leaf box $W$ is empty. The $X$ list consists of all boxes $A$ such that $B$ is in their $W$ lists. The FMM then consists of eight operators: P2M, P2L, M2M, M2L, L2L, L2P, M2P and the P2P, applied once to each applicable node over the course of two consecutive traversals of the octree (bottom-up and then top-down). The operators define interactions between a given ‘target’ node, and potentially multiple ‘source’ nodes from the tree. They are read as ‘X to Y’, where ‘P’ stands for particle(s), ‘M’ for multipole expansion and ‘L’ for local expansion. The direct calculation of a kernel interaction between the degrees of freedom contained in two boxes is referred to as the P2P operator. The non-adaptive case is similar, except the $W$ and $X$ lists are now empty.


\begin{algorithm}
\caption{Fast Multipole Method}
\label{alg:fmm:app_a_1}
\begin{algorithmic}

    \State $N$ is the total number of points
    \State $s$ is the maximum number of points in a leaf node.

    \State
    \State \textbf{Step 1: Tree construction}
    
    \For{each box $B$ in \textit{preorder} traversal of tree}
        \State subdivide $B$ if it contains more than $s$ points.
    \EndFor
    \For{each box $B$ in \textit{preorder} traversal of tree}
        \State construct \textit{interaction lists}, $U$, $V$, $X$, $W$
    \EndFor
    
    \State 
    \State \textbf{Step 2: Upward Pass}
    \For{each leaf box $B$ in \textit{postorder} traversal of the tree}
    \State \textbf{P2M}: compute multipole expansion for the particles they contain.
    \EndFor
    \For{each non leaf box $B$ in \textit{postorder} traversal of the tree}
    \State \textbf{M2M}: form a multipole expansion by translating and summing the expansion coefficients of the multipole expansions of its children.
    \EndFor

    \State
    \State \textbf{Step 2: Downward Pass}
    \For{for each non-root box $B$ in \textit{preorder} traversal of the tree}
    \State \textbf{M2L}: translate multipole expansions of boxes in $B$'s $V$ list to a local expansion at $B$.
    \State \textbf{P2L}: translate the charges of particles in $B$'s $X$ to the local expansion at $B$.
    \State \textbf{L2L}: translate $B$'s local expansion to its children.
    \EndFor 

    \For{each leaf box $B$ in \textit{preorder} traversal of the tree}
    \State \textbf{P2P}: Directly compute the local interactions between the particles in $B$ and its $U$ list.
    \State \textbf{L2P}: Translate local expansions for boxes in $B$'s $W$ list to the particles in $B$.
    \State \textbf{M2P}: Translate the multipole expansions for boxes in $B$'s $W$ list to the particles in $B$.
    \EndFor

\end{algorithmic}
\end{algorithm}

As mentioned above, the tree must be refined such that the leaf boxes contain only a small constant number
of particles, $s$. The maximum level of refinement is therefore approximately $n \approx \log(N)$, where $N$ is the number of particles in the tree. The multipole and local expansions are \textit{truncated} such that their expansion orders $p$, are chosen such that $p < N$,  the complexity of each translation operator (P2M, P2L, M2M, M2L, L2L, L2P, M2P) are bounded by complexities of the form $O(\kappa(p)N)$ where $\kappa(p)$ is a constant that depends on $p$, for all boxes in the tree. The direct calculations at the end are bounded by the maximum size of the $U$ lists, $|U|$, and $s$ as $O(s|U|N)$. The whole algorithm can therefore be seen to be bounded by $O(N)$. We defer to the literature for a more detailed analysis \cite{greengard1987fast}.