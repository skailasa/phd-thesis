
\thispagestyle{plain}

\begin{center}
    \textbf{Impact Statement}
\end{center}


This thesis establishes norms and practices for developing practical implementations of the kernel independent Fast Multipole Method (kiFMM), which will be of significant utility to the developers specialising in this and related algorithms. During this research we re-visited established codes for the kiFMM, identified software construction techniques that can lead to more flexible implementations that allow users to experiment, exchange, and build upon critical algorithmic subcomponents, computational backends, and problem settings - which are often missing from competing implementations which focus achieving specific benchmarks. For example, the flexibility of the software presented in this thesis allows for the critical evaluation of key algorithmic subcomponents, such as the `multipole to local' (M2L) operator which we presented in \cite{kailasa2024m2ltranslationoperatorskernel}.

As the primary outputs are open-source software libraries \cite{kailasa2022pyexafmm,kailasa2024kifmmrs} which are embedded within existing open-source efforts, most significantly the Bempp project, with an existing user-base the software outputs of this thesis are likely to have a wide ranging impact in academia and industry influenced by the demand for these softwares. Furthermore, the adoption and promotion of Rust for this project, and within our group, establishes further the utility of this relatively new language for achieving high-performance in scientific codes, which in recent years has been the subject of growing interest in the wider high-performance scientific computing community.
