\thispagestyle{plain}

\begin{center}
    \textbf{Abstract}
\end{center}

This thesis is concerned with the development of a software platform for the \acrfull{kifmm}, a variant of the widely applied \acrfull{fmm} algorithm which finds far reaching application across computational science. Indeed, for certain dense rank-structured matrices, such as those that arise from the boundary integral formulation of elliptic \glspl{pde}, the \acrshort{fmm} and its variants accelerate the computation of a matrix vector product from $O(N^2)$ to just $O(N)$ in the best case.

We demonstrate the efficacy of our software's flexible design by contrasting implementations of a key bottleneck known as the \acrfull{m2l} field translation, and present a new highly optimised approach based on direct matrix compression techniques and \acrshort{blas} operations, which we contrast with the current state of the art approach based on \glspl{fft} for \acrshort{kifmm}s. We show that we are able to achieve highly-competitive runtimes for three dimensional problems described by the Laplace kernel with respect to the state of the art, and often faster depending on the available hardware, with a simplified approach. Our approach is well suited to the direction of development of hardware architectures, and demonstrates the importance of re-considering the design of algorithm implementations to reflect underlying hardware features, as well as the enabling power of research software for algorithm development.

The software itself is written in Rust, a modern systems programming language, with features that enable a data oriented approach to design and simple deployment to common \acrshort{cpu} architectures. We describe our design and show how it allows us to extend our software to problems described by the  Helmholtz kernel, at low frequencies, as well as in a distributed memory setting, where emphasis has been placed on retaining a simple user interface and installation suitable for non-software experts, while remaining modular enough to remain open to open-source contribution from specialists. We conclude with both single node and \acrshort{hpc} benchmarks, demonstrating the scalability of our software as well as its state of the art performance.
