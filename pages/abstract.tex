\thispagestyle{plain}

\begin{center}
    \textbf{Abstract}
\end{center}

The past three decades have seen the emergence of so called `fast algorithms' that are able to optimally apply and invert dense matrices that exhibit a special low-rank structure in their off-diagonal elements. Such matrices arise in numerous areas of science and engineering, for example in the linear system matrices of boundary integral formulations of problems from acoustics and electromagnetics to fluid dynamics, geomechanics and even seismology. In the best case matrices can be stored, applied and inverted in $O(N)$, in contrast to $O(N^2)$ for storage and application, and $O(N^3)$ for inversion when computed naively.

The unification of software for the forward and inverse application of these operators in a single set of open-source libraries optimised for distributed computing environments is lacking, and is the central concern of this research project. We propose the creation of a unified solver infrastructure that can demonstrates good weak scaling from local workstations to upcoming exascale machines. Developing high-performance implementations of fast algorithms is challenging due to highly-technical nature of their underlying mathematical machinery, further complicated by the diversity of software and hardware environments in which research code is expected to run.

This subsidiary thesis presents current progress towards this goal. Chapter (\ref{chpt:1}) introduces the Fast Multipole Method (\gls{FMM}), the prototypical fast algorithm for $O(N)$ matrix vector products, and discusses implementation strategies in the context of high-performance software implementations. Chapter (\ref{chpt:2}) provides a survey of the fragmented software landscape for fast algorithms, before proceeding with a case study of a Python implementation of an FMM, which attempted to bridge the gap between a familiar and ergonomic language for researchers and achieving high-performance. The remainder of the chapter introduces Rust, our proposed solution for ergonomic and high-performance codes for computational science, and it concludes with an overview of a software output: Rusty Tree, a new Rust-based library for the construction of parallel octrees, a foundational datastructure for \gls{FMM}s, as well as other fast algorithms. Chapter (\ref{chpt:3}) introduces vectors for future research, specifically an introduction to fast algorithms and software for matrix inversion, and the potential pitfalls we will face in their implementation for performance, as well as an overview of a proposed investigation into the optimal mathematical implementation of field translations - a crucial component of a performant FMM. We conclude with a look ahead towards a key target application for our software, the solution of electromagnetic scattering problems described by Maxwell's equations that can demonstrate performance at exascale.
