\thispagestyle{plain}

\begin{center}
    \textbf{Abstract}
\end{center}

The past three decades have seen the emergence of so called `fast algorithms' that are able to optimally apply and invert dense matrices that exhibit a special low-rank structure in their off-diagonal elements. Such matrices arise in numerous areas of science and engineering, for example in the linear system matrices of boundary integral formulations of problems from acoustics and electromagnetics to fluid dynamics, geomechanics and even seismology. In the best case matrices can be stored, applied and inverted in $O(N)$, in contrast to $O(N^2)$ for storage and application, and $O(N^3)$ for inversion when computed naively. 

The unification of software for the forward and inverse application of these operators in a single set of open-source libraries optimised for distributed computing environments is lacking, and is the central concern of this research project. We propose the creation of a unified solver infrastructure that can demonstrates good weak scaling from local workstations to upcoming exascale machines. Developing high-performance implementations of fast algorithms is challenging due to highly-technical nature of their underlying mathematical machinery, further complicated by the diversity of software and hardware environments in which research code is expected to run.

This subsidiary thesis presents current progress towards this goal. We begin with overview of the fast algorithms of interest, and a summary of the goals and expected impact of this research in Chapter \ref{chpt:1:introduction}. Chapter \ref{chpt:2:python} details an early investigation of Python with Numba, an LLVM based `just-in-time' (JIT) compiler, as a tool for building out our software infrastructure. Despite its many advantages, including cross-platform support and large numerics ecosystem, we find Python and high-level languages in general to be inadequate for our purposes. In chapter \ref{chpt:3:rust} we propose Rust, a modern and fast developing systems programming language as our proposed solution for ergonomic and high-performance codes for computational science. In chapter \ref{chpt:4:current} we detail current open work streams in this research project, specifically current completed software outputs, computational investigations into optimal implementations of certain features of fast algorithms. We focus for now on the `fast multipole method' (FMM), an algorithm for the fast application of these special low-rank matrices to vectors, with the extension to fast-inverses which re-use much of the FMM's structure, planned as a future extension. Chapter \ref{chpt:conclusion} summarises our goals for the near, and medium term, as well as the expected outputs of this research.

