\thispagestyle{plain}

\begin{center}
    \textbf{Abstract}
\end{center}

Software has come to be a central asset produced during
computational science research. Projects that build off the research software outputs of external
groups rely on the software implementing proper engineering practice,
with code that is well documented, well tested, and easily extensible. As a
result research software produced in the course of scientific discovery has
become an object of study itself, and successful scientific software projects
that operate with performance across different software and hardware platforms,
shared and distributed memory systems, can have a dramatic impact on the
research ecosystem as a whole. Examples include projects such as the SciPy and
NumPy projects in Python, OpenMPI for distributed memory computing, or the package
manager and build system Spack, which collectively support a vast and diverse ecosystem of scientific
research.

This thesis is concerned with the development of a software platform
for so called `fast algorithms'. These algorithms have emerged in recent decades to
optimally apply and invert dense matrices that exhibit
a special low-rank structure in their off-diagonal elements. Such matrices arise
in numerous areas of science and engineering, for example in the linear system
matrices of boundary integral formulations of problems from acoustics and
electromagnetics to fluid dynamics, geomechanics and even seismology. In the
best case matrices can be stored, applied and inverted in $O(N)$, in contrast
to $O(N^2)$ for storage and application, and $O(N^3)$ for inversion when computed
naively. The diversity of the implementation approaches for these algorithms, as
well as their mathematical intricacy, makes the development of an ergonomic,
unified, software framework, that maximally re-uses data structures, is designed
for high performance, distributed memory environments, and works seamlessly
across platforms highly challenging. To date, many softwares for fast algorithms
are often written to benchmark the power of a particular implementation approach,
rather than with real user in mind.

 Modern programming environments are diverse, with unique trade-offs, in Chapter 1 we document our experience in transitioning from Python to Rust as a
modern, ergonomic, programming environment for our software. Chapter 2 introduces
fast algorithms in the context of current work streams, specifically the
implementation of a library for distributed memory octrees, and a distributed memory fast
multipole method (FMM) built on top of this. Chapter 3 documents active areas
of research and development for these projects, and proposed strategies for
outstanding issues. Finally, Chapter 4 looks ahead to the requirements for the
timely conclusion of this project as well as proposed extension areas.
