\thispagestyle{plain}

\begin{center}
    \textbf{Abstract}
\end{center}

Modern scientific computing is constrained by the dual requirement for practitioners to develop algorithms that exhibit optimal complexities and convergence, at the same time as remaining fast in practice, and malleable to new algorithmic and hardware advances. This thesis is concerned with the development of a software platform for the \acrfull{kifmm}, a variant of the widely applied \acrfull{fmm} algorithm which finds far reaching application across computational science. Indeed, for certain dense rank-structured matrices, such as those that arise from the boundary integral formulation of elliptic \acrfull{pdes}, the \acrshort{fmm} and its variants accelerate the computation of a matrix vector product from $O(N^2)$ to just $O(N)$ in the best case.

In this thesis we demonstrate the efficacy of our software's flexible design by contrasting implementations of a key bottleneck known as the \acrfull{m2l} field translation, and present a new highly optimised approach based on direct matrix compression techniques and BLAS operations, demonstrating its competitiveness with the current state of the art approach based on \acrfull{ffts} for \acrshort{kifmm}s. We show that we are able to achieve similar runtimes for three dimensional problems described by the Laplace kernel to the state of the art, and often faster depending on the available hardware, with a simpler approach. The introduced approach is well suited to the direction of development of hardware architectures, and demonstrates the importance of re-considering the design of algorithm implementations to reflect underlying hardware features.

The software itself is written in Rust, a modern systems programming language, with features that enable a `data oriented’ approach to design, where operations are centred on minimising memory movements. We describe our approach to the design of the software, the flexibility of which allows us to extend our software to problems described by the Helmholtz kernel, at low frequencies, as well as in a distributed memory setting.
