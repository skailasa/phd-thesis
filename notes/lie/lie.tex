\documentclass[12pt, a4, twoside]{article}
\usepackage[utf8]{inputenc}
\usepackage{graphicx}
\usepackage{algorithm}
\usepackage{algpseudocode}
\usepackage{amsmath}
\usepackage{amsfonts}
\usepackage{hyperref}
\usepackage{mathtools}
\DeclarePairedDelimiter\ceil{\lceil}{\rceil}
\DeclarePairedDelimiter\floor{\lfloor}{\rfloor}

\hypersetup{
    colorlinks=true,
    linkcolor=blue,
    filecolor=magenta,
    urlcolor=cyan,
}


\DeclareMathOperator\hmat{\mathcal{H}}
\DeclareMathOperator\htwomat{\mathcal{H}^2}
\DeclareMathOperator\grad{\text{grad}}
\DeclareMathOperator\vdiv{\text{div}}


\usepackage[backend=bibtex]{biblatex}
\addbibresource{lie.bib}

\title{Linear Integral Equations}
\author{Srinath Kailasa \thanks{srinath.kailasa.18@ucl.ac.uk} \\ \small University College London}

\date{\today}

\begin{document}
\maketitle

These notes are based on chapters from \cite{Kress2014}. They are meant as a summary, and generally exclude rigorous/long proofs or demonstrations, pointing the reader (me) to the correct place in the text.

\subsection{Chapter 6 - Potential Theory}

Concerned in this section with \textit{harmonic functions}. Green's theorems are essential to their study.

\textbf{Green's First Theorem}. Let $D$ be a bounded domain of class $C^1$, and let $\nu$ denote the outerward unit normal to a boundary $\partial D$ directed to the exterior of $D$. Then for $u \in C^1(\bar{D})$ and $v \in C^2(\bar{D})$:

\begin{flalign}
\int_{D} (u \Delta v + \grad u \cdot \grad v) dx = \int_{\partial D} u \frac{\partial v}{\partial \nu} ds
\end{flalign}\label{eq:green:1}

This theorem relates a volume integral through $D$ to a surface integral over its boundary. It's essentially a generalisation of the integration by parts formula.

Using \textbf{Gauss' divergence theorem}

\begin{flalign}
    \int_D \div A dx = \int_{\partial D} \nu \cdot A ds
\end{flalign}

where $A \in C^1(\bar{D})$ is a vector field defined by $A := u \grad v$, with the identity,

$$ \vdiv(u \grad v) = \grad u \cdot \grad v + u \vdiv \grad v$$

We substitute this identity to find (\ref{eq:green:1}). For \textbf{Green's second theorem},

\begin{flalign}
\int_D (u \Delta v - v \Delta u)dx = \int_{\partial D} \left ( u \frac{\partial v}{\partial \nu} -v \frac{\partial u}{\partial \nu} \right )
\end{flalign}\label{eq:green:2}

For $u, v \in C^2(\bar{D})$. We interchange $u$ and $v$ in the identity, apply it to the divergence theorem, and subtract the quantities.

Divergence theorem basically relates the flux of a vector field through a closed surface, to its divergence in the enclosed volume.

For $v \in C^2(\bar{D})$, harmonic in $D$, Then

\begin{flalign}
    \int_{\partial D} \frac{\partial v}{ \partial \nu} ds = 0
\end{flalign}

This follows from choosing $u=1$ in (\ref{eq:green:1}). The following formula known as \textbf{Green's Formula} is incredibly important. It allows us to calculate a harmonic function $u \in C^2(\bar{D})$ from its boundary values/derivatives (Cauchy data) as well as the fundamental solution,

\begin{flalign}
    u(x) = \int_{\partial D} \left \{ \frac{\partial u}{\partial \nu} \Phi(x, y) - u(y) \frac{\partial \Phi(x, y)}{\partial \nu (y)} \right \} ds(y), \> \> x \in D
\end{flalign}

See p. 77 of \cite{Kress2014} for a proof. Harmonic functions are analytic, ie each harmonic function has a local power series expansion.

Well posedness of BVP - a unique solution depending continuously on the given boundary data.



\end{document}