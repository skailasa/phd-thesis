\section{Emerging Developments}\label{chpt:1:sec:3}

Despite the above criticisms, high-level languages as tools for high-performance scientific computing remain an intense area of research and development. `Mojo' is a new programming language, along with a compiler. It's built as a superset of Python, specifically with the two-language problem in mind. Additionally, it attempts to address the `three language problem', whereby languages also target exotic hardware such as GPUs and TPUs \cite{Lattner2023Mojo}.

Led by a team that includes the original developers of LLVM, Mojo aims to simplify the development of high-performance applications in a Python-like language, that acts as a superset of Python. Moreover, it seeks to make these applications deployable across most hardware and software targets, ensuring compatibility with Python's vast open-source libraries and straightforward build tools.

This is achieved by building on the MLIR compiler infrastructure. MLIR can be thought of as a generalisation of LLVM, catering to CPUs, GPUs, and novel ASICs for AI. The team chose to develop around Python to leverage its extensive existing user base in computational and data sciences.

Currently, Mojo remains a closed-source language and is actively being developed by its parent company, Modular. Thus, even though it appears promising, it's not yet in a state suitable for experimentation. Nevertheless, Mojo showcases the potential of a future programming environment that might definitively `solve' the problems developers face when selecting a programming environment for academic software.