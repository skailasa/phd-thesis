\section{Developing Scientific Software}\label{chpt:1:sec:0}

Scientific software development presents a unique set of challenges. Although development teams are frequently small, they are tasked with producing highly optimised code that must be deployed across a myriad of hardware and software platforms. Moreover, there is a pressing need for comprehensive documentation and rigorous testing to ensure reproducibility. Given that many of these softwares arise within doctoral programs or other short-term projects, there is a tendency to tailor software development to showcase a specific project's objectives. Whether that be to demonstrate a convergence result of a specific methodological improvement, or offer a new benchmark implementation of an algorithm. Consequently, once the principal results are achieved these software projects often become orphaned, lack compatibility with a range of development platforms, or aren't adaptable to related challenges and subsequent research by other teams.

A recent survey of 5000 software tools published in computational science papers featured in ACM publications found that repositories for computational science papers had a median active development span of a mere 15 days. Alarmingly, one third of these repositories had a life cycle of less than one day \cite{hasselbring2020open}. Implying that upon the publication of the affiliated paper, software is typically abandoned or, at best, receives private maintenance. This trend underscores the challenge of dedicating sustained resources in an academic setting to software upkeep, even when such maintenance is vital for reproducibility. It may also hint at a deficiency in professional software engineering expertise among computational researchers whose principal expertise lies elsewhere.

Therefore, confronting the challenge of developing maintainable research software relies on the choice of programming environment. Developers need to have a frictionless system for testing, documenting, using existing open-source solutions and building extensions to their code. Software design has to be general enough to extend to new algorithmic developments, but also malleable enough for external developers and users to adapt software to new usecases as well as their own needs. Building software for diverse software environments and target architectures should also be painless as possible to encourage large-scale adoption. Additionally, domain scientists who are typically not experienced in low-level software development require interfaces to familiar high level languages, which must be easy to maintain for core-developers.

In the early stages of this research project we experimented with Python, a high-level interpreted language, that has become a de-facto standard in data-science and numerical computing for a wide variety of domain scientists. Recent years have seen the development of tools that allow for the compilation of fast machine-code from Python, allowing for multi-threading, and the targeting of both CPU and GPU architectures \cite{lam2015numba}. This approach takes advantage of the LLVM compiler infrastructure for generating fast machine code from Python via the Numba library, and is similar to other approaches to creating fast compiled code from high-level languages such as Julia \cite{bezanson2017julia}. We built a prototypical single-node multithreaded implementation of a fast algorithm, the fast multipole method (FMM), in Python to test the efficacy of this approach. However, we found that for complex algorithms writing performant Numba code can be challenging, especially when performance relies on low-level management of memory \cite{kailasa2022pyexafmm}. We summarise this experience in section \ref{chpt:1:sec:1}. We identified Rust, a modern low-level compiled language, as a promising programming environment for our software. Rust has a number of excellent features for scientific software development, most notably the introduction of a `borrow checker', that enforces the validity of memory references at compile time preventing the existence of data races in compiled Rust code, as well as its runtime `Cargo', which offers a centralised system for dependency management, compilation, documentation and testing of Rust code. We summarise Rust's benefits, as well as notable constraints, in section \ref{chpt:1:sec:2}. We conclude this chapter by noting that language and compiler development for scientific computing is an active area of research, in section \ref{chpt:1:sec:3} we contrast Rust with emerging programming environments for scientific software.