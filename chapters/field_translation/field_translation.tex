\chapter{Performant Multipole To Local Field Translation for the kernel independent Fast Multipole Method}\label{chpt:field_translation}
\thispagestyle{chaptertitle} % Force the fancy style on this page

\begin{center}
    \textit{The discussion in this chapter, including figures and diagrams, is adapted from the material first presented in \cite{kailasa2024m2ltranslationoperatorskernel} }
\end{center}


\section{Forming the Multipole To Local Translation for Kernel Independent Fast Multipole Methods}

\section{Computational Constraints}

- Exactly why is this hard to evalaute?
- Maybe complexity estimates of memory accesses.
- Layout discussion.


\section{Review of Acceleration Methods}

- Full literature review of past approaches
- brief summary of analytical approaches (1-2 paragraphs)
- Where past efforts have been focussed, and why? (Original paper dismissed direct matrix compression)
- Why this may be a good idea now (randomised methods, available hardware)

- use this section to introduce idea of transfer vectors, reflection and rotational symettry
- how this is achieved in practice (i.e. what computations are needed, not the implementation details)

Fast Multipole Method as a Matrix-Free Hierarchical Low-Rank Approximation
Rio Yokota, Huda Ibeid, David Keyes


\section{A Direct Matrix Compression Based Acceleration Scheme}

- Why might this be preferred, or advantageous, what are its constraints
- Why it is counterintuitive


\section{FFT based Acceleration Based Schemes}

- Explanation of the method, and why it was able to achieve high performance.
- Why this may not be completely appropriate, low arithmetic intensity (maybe estimate?)


- Both algorithmic and computational

- Review of approaches, success and failures, and what works in the context of modern software and hardware systems.

- Can include section reviewing PVFMM approach

- Approaches for BLAS based field translation in some more detail than in the paper.

- Our approach, and why it works with reference to the software and hardware currently available.

- Articulate the significance of this result

- Why our approach is sustainable given long term trends in hardware and software.

- Why might it be useful for Helmholtz FMM ...

- What are important trends, and what have we actually done.

- Interesting point of comparison with ScalFMM to demonstrate the importance of caching to performance, noting that direct software comparisons are not entirely fair.

