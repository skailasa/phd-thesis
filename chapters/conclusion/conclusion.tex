\chapter{Conclusion}\label{chpt:conclusion}

In this subsidiary thesis we've presented progress on the development of a new software infrastructure for fast algorithms. We've documented recent outputs towards this goal including foundational software as well as algorithmic techniques. The main outputs being an investigation into programming languages and environments most suitable for scientific computing, investigations to ensure an ergonomic design for our software, a distributed load balanced octree library designed for high-performance, as well as significant inroads to a distributed FMM based on this by studying sparsification schemes for the multipole-to-local translation operator $T^{M2L}$.

The immediate next steps of this project will be to publish our recent software results on octrees and the parallel FMM in an appropriate scientific journal, and release a first version of our software. The final stages of this project will focus on completing the outlined improvements to our translation operator library to achieve, and hopefully supersede the current state of the art, creating a new benchmark distributed FMM library that is open to extension to other fast algorithms.