\chapter{The FMM for HPC}\label{chpt:distributed_fmm}
\thispagestyle{chaptertitle} % Force the fancy style on this page

- Focus is on the maximal reduction in communication.

- what communication can and cannot be avoided?

- How the local/global split in terms of tree gives rise to optimal communication scheme.

- What simplifying assumptions can we take for most pre-exascale systems?

- Avoid sorting of Morton keys/point data, the local/global split gives us a way to statically partition tree across available resources - simplifying assumption if work with ncpu = pow(8).
- Not restricted to this, but makes threading simpler for local FMMs.

- Optimal implementation of MPI primitives for common data sizes.

- What will probably not work approaching exascale?
- the gather operation over all processes required for multiple steps of this algorithm - ghost exchange, multipoles at local root on nominated processor. How can these problems be addressed? Do they even matter for the problem sizes we're concerned with?

