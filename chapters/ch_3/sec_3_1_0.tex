As noted in Chapter \ref{chpt:2:sec:2}, the $T^{M2L}$ is the most computationally intensive phase of an FMM implementation. In this section we describe two major approaches to accelerate its computation, chiefly a numerical compression scheme based on taking an SVD and relying on the low-rank nature of $T^{M2L}$, and secondly an `exact' method that relies on an FFT to redue the complexity of the convolution represented by $T^{M2L}$. We have implemented both via our software `Bempp-Field'\footnote{https://github.com/bempp/bempp-rs/tree/main/field}, and we use this section to describe the relative merits of these approaches, implementation challenges, as well as open questions which remain. We find that the FFT approach, which has been demonstrated to perform well in single node \cite{wang2021exafmm} as well as multi node \cite{malhotra2015pvfmm} to be our favoured approach. We propose new algorithm for its implementation, and offer some benchmarks for its performance in a single-node setting relative to the state of the art \cite{wang2021exafmm,malhotra2015pvfmm} which rely on a similar approach.


