
\chapter{Review of Fast Multipole Methods}\label{chpt:fmm}
\thispagestyle{chaptertitle} % Force the fancy style on this page

\section{Kernel Independent Fast Multipole Methods}

- Review of the KiFMM and variants. Black Box FMM, Analytical FMM, Data Driven Techniques.

- Motivation for use from a software engineering and computational performance perspective.

- Data flow during the KiFMM.

- Performance characteristics and features of the kiFMM.

- Reflection on the kiFMM and modern software and hardware

The decades since its original presentation have seen the \acrshort{fmm} extended with related ideas which principally differ in how they represent interactions between distant clusters of particles typically referred to as \textit{algebraic} FMMs.


- Black Box vs Analytical

\section{Oscillatory Fast Multipole Methods}

- helmholtz fmm

- high frequency helmholtz FMM, out of scope of the thesis but can mention that they exist. Especially in the context of kiFMM approaches, what is the key difference? When does it apply (the directional low rank condition), where is there an additional loop? Why might this result in greater complexity

\section{Related Ideas}

- H Matrix and H2 matrices, and wider setting of the FMM and related problems.

- Abduljabbar thesis contains a nice summary I can read.


\section{The Fast Multipole Method's Computational Structure}

- Parallelism levels in computing (ILP (Pipelining, Superscalar, Speculative execution), Data level (SIMD, GPU), Thread level TLP (multithreading, simultaneous multithreading and hyper threading), Process level (symettric and asymettrixc multitprocessing), task level, Distributed Parallelism e.g. MPI and MapReduce)

- Only some of these are relevant for scientific computing

- Examine FMM data flow and relate to levels of Parallelism and which will be taken advantage of by us, and which are yet to be examined.

- What is the trend in hardware and why is the FMM a good kernel for scaling in future computer systems?

- What are the principal difficulties we will encounter? Data organisation, and communication costs in a distributed setting.

- What about good FMM software? Specialised kernels and substructures are required to be generically interfaced.

- What parts of this are addressed by this thesis and where?


\section{Review of Software Approaches}

- What software exists, and which approaches do they use?

- What are they optimised for, and what kind of performance do they promise?

- What are the trade-offs of each software

- Hardware targetted by each available software, what's missing?

- What's available, and what are the shortcomings?

\section{Thesis Structure}\label{chpt:fmm:sec:layout}


In Chapter \ref{chpt:fmm} we perform a literature review of methods and software for modern \acrlong{fmm}s, with a specific focus on so called `kernel independent' or `black box' \acrshort{fmm}s in Sections ... and ... which are the focus of our implementation efforts. We review related ideas which share many features of the \acrshort{kifmm}s in Section ..., such as the $\mathcal{H}$ and $\mathcal{H}^2$ matrix approaches. We move on to a review of the \acrshort{kifmm}s computational structure in Section ..., where we provide estimates of the computational complexities of its operators, and identify the parallelism available in the algorithm with respect to that provided by modern hardware. We conclude with a review of past software efforts for \acrshort{fmm}s, and place our contribution within this context.


A major effort of this thesis was designing a \textit{platform} for \acrshort{kifmm}s. Whereby, one is free to experiment with the implementation of subcomponents in a highly modular way, while retaining performance and the use of the remainder of the library. Therefore a significant early investigation was into appropriate tooling environments for scientific software, first presented in \cite{kailasa2022pyexafmm}. We present this investigation in Chapter \ref{chpt:programming_for_science}, where we document our experience with Python as an alternative for achieving low-level performance as well as our chosen platform Rust, a relatively new language emerging as a contender for performant and productive research software.

Chapter \ref{chpt:field_translation} details a rigorous application of our framework, where we investigated optimisations for the crucial \acrshort{m2l} field translation, recently presented in \cite{kailasa2024m2ltranslationoperatorskernel}. We find non-intuitively that direct matrix compression techniques for admissable blocks can be highly competitive with state of the art optimal schemes based on \acrshort{ffts} for three dimensional problems described by the Laplace kernel.

Chapter \ref{chpt:software_design} describes in detail the engineering approach of our software, particularly the employment of Rust's trait system, as well as specific implementation details of the \acrshort{kifmm}s operators. In Chapter \ref{chpt:hpc} we discuss the design and implementation of our software framework for distributed memory systems, detailing communication reducing schemes for the communication of ghost information.

Chapter \ref{chpt:experiments} contains numerical experiments with our software in a single node (Section ...) as well as HPC (Section ...) setting, including a study of the applicability of our software to problems described by Helmholtz problems with low to moderate wavenumbers.


We conclude with a reflection on our results and suggestions for future investigations in Chapter \ref{chpt:conclusion}.

