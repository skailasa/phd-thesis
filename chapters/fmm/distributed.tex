
\section{Distributed Memory Fast Multipole Methods}\label{chpt:fmm:sec:distributed}

Supercomputers are usually defined as massively parallel machines characterised by hundreds to thousands of individual nodes comprising of individual hardware pieces which communicate via a network. In this context memory movements, now between compute nodes, are of even greater practical relevance

- Network topologies, NVLink and other RDMA technologies that are emerging.
- Rate of improvement in bandwidth/latency on network vs DRAM.

Minimising communication is crucial
- Ibeid communication costs, and how these can be simplified.




- How are FMMs distributed for distributed memory?

- Focus is on the maximal reduction in communication.

- what communication can and cannot be avoided?

- How the local/global split in terms of tree gives rise to optimal communication scheme.

- What simplifying assumptions can we take for most pre-exascale systems?

- Avoid sorting of Morton keys/point data, the local/global split gives us a way to statically partition tree across available resources - simplifying assumption if work with ncpu = pow(8).
- Not restricted to this, but makes threading simpler for local FMMs.

- Optimal implementation of MPI primitives for common data sizes.

- What will probably not work approaching exascale?
- the gather operation over all processes required for multiple steps of this algorithm - ghost exchange, multipoles at local root on nominated processor. How can these problems be addressed? Do they even matter for the problem sizes we're concerned with?

- Bandwidth and latency complexity estimates for communication approaches
- Bandwidth
    - total data transfer, transfer per process, scaling with PARFOR
- latency
    - Number of communication steps,
    - scaling with P
