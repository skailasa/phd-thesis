\chapter{Numerical Experiments}\label{chpt:experiments}
\thispagestyle{chaptertitle} % Force the fancy style on this page


\section{Laplace}\label{experiments:sec:laplace}
\subsection{Single Node}\label{chpt:experiments:sec:laplace:sub:single_node}

\subsection{Multi Node}\label{chpt:experiments:sec:laplace:sub:multi_node}

- Weak scaling, communication vs computation time breakdown.

- Load balance discussion, does it matter for the distributions tested?

- Discussion on impacts of bandwidth and latency, and potential for async.

- Trade-off sort methods.


\section{Helmholtz}\label{chpt:experiments:sec:helmholtz}

\subsection{Single Node}\label{chpt:experiments:sec:helmholtz:sub:single_node}

Simple benchmark for low $k$

- A graph of $p$ vs level for each given accuracy, e.g. pick a few in single and double, likely to be easiest and most appropriate for single precision, and low double precision.

- Big colorful plot of HF helmholtz, use same parameters as in the Lexing Ying/Engquist paper, can't directly compare runtimes - but these are considered high frequencies in 3D.

- Comment, with optimal $K$ evaluations appropriately using SIMD, relatively shallow trees, especially in single precision, can model very high frequency problems in practically useful runtimes, though perhaps (need to check) lose asymptotic scaling.


