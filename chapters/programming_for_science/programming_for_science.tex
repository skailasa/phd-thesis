\chapter{Modern Programming Environments for Science}\label{chpt:programming_for_science}
\thispagestyle{chaptertitle} % Force the fancy style on this page



\begin{center}
    \textit{The discussion in this chapter, including figures and diagrams, is adapted from the material first presented in \cite{kailasa2022pyexafmm}.}
\end{center}

\section{Requirements for Research Software}

- Requirements and constraints on research software development.

- As an example FMM softwares used in recent benchmark studies (ExaFMM variants, PVFMM) have been constructed during the course of doctoral or post-doctoral projects. This entails a significant `key man' risk, in which when the project owner completes their course of research the project enters a decay state and is no longer actively maintained and developed. New developers, unfamiliar with the code bases which can grow to thousands of lines of code, and often written without reference to standard software engineering paradigms for designing and managing large code bases (continuous integration, software diagrams, and simple decoupled interfaces) will find it challenging to build upon existing advances, and resort to developing new code-bases from scratch, rediscovering implementation details that are often critical in achieving practical performance.

- In seeking to avoid this cycle we envisioned a project built in Python, which maximises the maintainability of a project due to its simple syntax and language construction. New developers who, as a standard, are often educated in Python in the natural sciences and engineering, will hopefully be familiar with the language in order to gain productivity as fast as possible. However, as we demonstrated in our paper ... This itself imposes significant constraints on performance, which is balanced by the 'usability' of the language, making it just as challenging as developing a complex code in a compile language.

- Modern compiled languages offer tools that enable developer productivity. Examples include Go, Rust, ...

- The complexity of methods leads to complex code surface areas which are difficult to maintain especially in an academic setting with few resources for professional software engineering practice.

- The diversity of hardware and software backends leads to increasing difficulty for projects to experiment with and incorporate computational advances.

- Hardware and software complexity, and gap between a one-off coding project and extensible maintainable software tooling.

- Review developments in computer hardware and software that make this easier to be more productive, but also more challenging to wrap together over time.

- Emerging and future trends, exemplified by the step change in compiled langauges in the new generation and the interest in Rust and similar langauges. The mojo project and what this says about the future.


\section{Low Level or High Level? Balancing Simplicity with Performance}

- Summary of Python paper results, in summary complex algorithms necessitate complex code in order to achieve performance - specifically the requirement for programmers to be in charge of memory and for hot sections manually vectorise etc. Writing everything in a high-level language obfuscates the application code from the sections critical to performance

- Review of why this was thought to be a good idea, and why it might be worth trying again in the future.

- What problems does this paper address, wrt to the literature?

- Brief review of motivation and reasoning behind Rust, and which features we take advantage of

- Build system, paragraph explaining it and contrast with build systems wrt to competitors.

- Trait system, though shares similarities with C++21 concepts, a part of the type systems
    - organisation of shared behaviour without loss of performance, bottom up organisation.



- Why the two language problem isn't really a problem with modern compiled langauges, note on the succesful projects that manage this and how effective just writing C interfaces can be, if using another language.


