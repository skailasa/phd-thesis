\chapter{Modern Programming Environments for Science}\label{chpt:programming_for_science}
\thispagestyle{chaptertitle} % Force the fancy style on this page



\begin{center}
    \textit{The discussion in this chapter, including figures and diagrams, is adapted from the material first presented in \cite{kailasa2022pyexafmm}.}
\end{center}

\section{Requirements for Research Software}

- Requirements and constraints on research software development.

- As an example FMM softwares used in recent benchmark studies (ExaFMM variants, PVFMM) have been constructed during the course of doctoral or post-doctoral projects. This entails a significant `key man' risk, in which when the project owner completes their course of research the project enters a decay state and is no longer actively maintained and developed. New developers, unfamiliar with the code bases which can grow to thousands of lines of code, and often written without reference to standard software engineering paradigms for designing and managing large code bases (continuous integration, software diagrams, and simple decoupled interfaces) will find it challenging to build upon existing advances, and resort to developing new code-bases from scratch, rediscovering implementation details that are often critical in achieving practical performance.

- In seeking to avoid this cycle we envisioned a project built in Python, which maximises the maintainability of a project due to its simple syntax and language construction. New developers who, as a standard, are often educated in Python in the natural sciences and engineering, will hopefully be familiar with the language in order to gain productivity as fast as possible. However, as we demonstrated in our paper ... This itself imposes significant constraints on performance, which is balanced by the 'usability' of the language, making it just as challenging as developing a complex code in a compile language.

- Modern compiled languages offer tools that enable developer productivity. Examples include Go, Rust, ...

- The complexity of methods leads to complex code surface areas which are difficult to maintain especially in an academic setting with few resources for professional software engineering practice.

- The diversity of hardware and software backends leads to increasing difficulty for projects to experiment with and incorporate computational advances.

- Hardware and software complexity, and gap between a one-off coding project and extensible maintainable software tooling.

- Review developments in computer hardware and software that make this easier to be more productive, but also more challenging to wrap together over time.

- Emerging and future trends, exemplified by the step change in compiled langauges in the new generation and the interest in Rust and similar langauges. The mojo project and what this says about the future.


\section{Low Level or High Level? Balancing Simplicity with Performance}

- Summary of Python paper results, in summary complex algorithms necessitate complex code in order to achieve performance - specifically the requirement for programmers to be in charge of memory and for hot sections manually vectorise etc. Writing everything in a high-level language obfuscates the application code from the sections critical to performance

- Review of why this was thought to be a good idea, and why it might be worth trying again in the future.

- What problems does this paper address, wrt to the literature?

- Brief review of motivation and reasoning behind Rust, and which features we take advantage of

- Review of data oriented design, how this can be enabled with traits.

% \section{Developing Scientific Software}\label{chpt:1:sec:0}

Scientific software development presents a unique set of challenges. Although development teams are frequently small, they are tasked with producing highly optimised code that must be deployed across a myriad of hardware and software platforms. Moreover, there is a pressing need for comprehensive documentation and rigorous testing to ensure reproducibility. Given that many of these softwares arise within doctoral programs or other short-term projects, there is a tendency to tailor software development to showcase a specific project's objectives. Whether that be to demonstrate a convergence result of a specific methodological improvement, or offer a new benchmark implementation of an algorithm. Consequently, once the principal results are achieved these software projects often become orphaned, lack compatibility with a range of development platforms, or aren't adaptable to related challenges and subsequent research by other teams.

A recent survey of 5000 software tools published in computational science papers featured in ACM publications found that repositories for computational science papers had a median active development span of a mere 15 days. Alarmingly, one third of these repositories had a life cycle of less than one day \cite{hasselbring2020open}. Implying that upon the publication of the affiliated paper, software typically gets abandoned or, at best, receives private maintenance. This trend underscores the challenge of dedicating sustained resources in an academic setting to software upkeep, even when such maintenance is vital for reproducibility. It may also hint at a deficiency in professional software engineering expertise among computational researchers whose principal expertise lies elsewhere.

Therefore, confronting the challenge of developing maintainable research software relies on the choice of programming environment. Developers need to have a frictionless system for testing, documenting, using existing open-source solutions and building extensions to their code. Software design has to be general enough to extend to new algorithmic developments, but also malleable enough for external developers and users to adapt software to new usecases as well as their own needs. Building software for diverse software environments and target architectures should also be painless as possible to encourage large-scale adoption. Additionally, domain scientists who are typically not experienced in low-level software development require interfaces to familiar high level languages, which must be easy to maintain for core-developers.

In the early stages of this research project we experimented with Python, a high-level interpreted language, that has become a de-facto standard in data-science and numerical computing for a wide variety of domain scientists. Recent years have seen the development of tools that allow for the compilation of fast machine-code from Python, allowing for multi-threading, and the targeting of both CPU and GPU architectures \cite{lam2015numba}. This approach takes advantage of the LLVM compiler infrastructure for generating fast machine code from Python via the Numba library, and is similar to other approaches to creating fast compiled code from high-level languages such as Julia \cite{bezanson2017julia}. We built a prototypical single-node multithreaded implementation of a fast algorithm, the fast multipole method (FMM), in Python to test the efficacy of this approach. However, we found that for complex algorithms writing performant Numba code can be challenging, especially when performance relies on low-level management of memory \cite{kailasa2022pyexafmm}. We summarise this experience in section \ref{chpt:1:sec:1}. We identified Rust, a modern low-level compiled language, as a promising programming environment for our software. Rust has a number of excellent features for scientific software development, most notably the introduction of a `borrow checker', that enforces the validity of memory references at compile time preventing the existence of data races in compiled Rust code, as well as its runtime `Cargo', which offers a centralised system for dependency management, compilation, documentation and testing of Rust code. We summarise Rust's benefits, as well as notable constraints, in section \ref{chpt:1:sec:2}. We conclude this chapter by noting that language and compiler development for scientific computing is an active area of research, in section \ref{chpt:1:sec:3} we contrast Rust with emerging programming environments for scientific software.
% - Summary of the algorithm's logic - and the low rank assumption behind its power. 

- Overview of the choices that can be made: translations, expansions, etc.

~ 2 pages
% \section{Introducing Rust for Scientific Software}\label{chpt:1:sec:2}


% \section{Emerging Developments}\label{chpt:1:sec:3}

Despite the above criticisms, high-level languages as tools for high-performance scientific computing remain an intense area of research and development. `Mojo' is a new programming language, along with a compiler. It's built as a superset of Python, specifically with the two-language problem in mind. Additionally, it attempts to address the `three language problem', whereby languages also target exotic hardware such as GPUs and TPUs \cite{Lattner2023Mojo}.
    
This is achieved by building on the MLIR compiler infrastructure. MLIR can be thought of as a generalisation of LLVM, catering to CPUs, GPUs, and novel ASICs for AI. The team chose to develop around Python to leverage its extensive existing user base in computational and data sciences.

Currently, Mojo remains a closed-source language and is actively being developed by its parent company, Modular. Thus, even though it appears promising, it's not yet in a state suitable for experimentation. Nevertheless, Mojo showcases the potential of a future programming environment that might definitively `solve' the problems developers face when selecting a programming environment for academic software.
